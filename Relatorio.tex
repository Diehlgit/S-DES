\documentclass[12pt]{article}

\title{S-DES}
\author{Gabriel Henrique do Nascimento Neres \\ Arthur Diehl Barroso}
\date{\today}

\begin{document}
\maketitle

\begin{abstract}
  Texto detalhando o desenvolvimento de uma versão simplificada do algoritmo de criptografia DES, denominado de S-DES, que realiza a encriptação e decriptação utilizando dois modos de operação, o ECB e CBC. Nele é utilizado uma chave de 10 bits e uma entrada de convertida em uma lista de bytes. 
  \textbf{Palavras-chave}: Criptografia, DES, ECB, CBC.
\end{abstract}

Esse trabalho irá implementar e detalhar as etapas de confecção do algoritmo de criptografia simetria S-DES, um modelo simplificado do algoritmo DES. Para compreender melhor a implementação que está sendo feita, serão repassados os resultados de cada função baseado nos dados de entrada a baixo:

\begin{itemize}
  \item Chave: 1010000010
  \item Mensagem: 11010111 01101100 10111010 11110000
  \item IV CBC: 01010101
\end{itemize}

\section{Geração de Chaves Subjacentes}
No processo de geração das chaves subjacentes os 10 bits iniciais são transformados em 2 sub-chaves (k1 e k2) derivadas a partir da chave recebida. Para obter as sub-chaves são utilizadas 2 funções principais em conjunto com um array que representa as permutações.

Bloco de código:

def permutation(s: str, indices: list[int]) -> str:
    return ''.join(s[i] for i in indices)

\section{Modos de operação}
Ao iniciar o algoritmo, poderão ser utilizados dois modos de operação. Esses algoritmos são:

\subsection{ECB}
A técnica utilizada por esse modo de operação é bem simples, principalmente pela entrada fornecida será dividida em entradas do mesmo tamanho da utilizada pelo SDES, ou seja, a entrada do ECB será de um conjunto de blocos de 8 bits que serão passados um a um para o algoritmo de encriptação do SDES.

Para isso foi utilizada a função:

IMAGEM ECB ENCRYPT

Para exemplificar a entrada fornecida ao programa ao passar pela função será convertida ao array [11010111, 01101100, 10111010, 11110000]. Vale lembrar que esse \textbf{não} é o valor retornado pela função, pois este só será possível de obter ao finalizar todo o restante do algoritmo do SDES.


\subsection{CBC}

\section{Permutação Inicial}
Para realizar a permutação inicial é utilizada a mesma função de permutação utilizada na geração das sub-chaves. O valor da permutação é definido pelo array [1, 5, 2, 0, 3, 7, 4, 6] e terá como retorno os seguintes valores em cada um dos modos de operação:


\section{Divisão em Metades}

\section{Rodades de Feistel}

\section{Permutação Final}

\end{document}